\section{Introduction and construction}\label{dec:introduction}
\label{sec:basic_coupling}

\begin{quote}
{\small In this section we construct the East-process and the 1-sided contact process on the same probability space using the graphical method. This construction gives rise to an important monotonicity property that we use extensively in later sections. }
\end{quote}

\subsection{Introduction}\label{ssec:introduction}
The East-process is an interacting particle system evolving with a Glauber like dynamics on the state space $\Omega = \{0,1\}^\Z$. It is part of a class of stochastic processes called kinetically constrained spin models (KCMs), with the East-process being the first of these to be studied rigorously. The process evolves as follows: at each site $x \in \Z$ the system tries to update the value of the spin at $x$ to 1 or 0 at rate $p \in (0,1)$ and $q \defeq 1 - p$ respectively. The update is accepted only if the local constraint is satisfied, which in the East-processes case is that the spin value at site $x-1$ must be equal to 1. Sometimes we will call elements of $\Omega$ \textit{configurations} and say a site is \textit{occupied} or \textit{infected} if its spin value is equal to 1. \\

In the sections to follow we focus on two objects of interest related to the East-process. The first one is the speed of the so-called \textit{front}. Consider an East-process started from the configuration equal to all 0 except at the origin. It is easy to see that the spins on $(\infty, 0]$ stay frozen for all time, and infection 'spreads' to the right. A natural question to ask then how fast this spreading of infection happens if it happens at all. The front at time $t$ of this process is defined as the rightmost infected site in the configuration at time $t$. We will show that for large enough $p$ the front propagates at exactly linear speed. \\

The second object of interest is the mixing time of the East process when restricted to $\{ 1, 2, ..., L\}$ for some $L \in \N^+$. We will study the mixing time for the East-process on $\{ 1, 2, ..., L\}$ with a 1 fixed at the origin, so that the evolution at site 1 is unconstrained and go on to show that for large enough $p$ the mixing time is $\Omega(L)$.  \\

In our study of the speed of the front we will compare the East-process to a second stochastic process called the 1-sided contact process on $\Z$. The 1-sided contact process on $Z$ has the same state space $\Omega$ and evolves as follows: each site infects its neighbour to the right at rate $p$ and 'recovers' i.e. sets its own spin to 0 at rate $q$. 

\subsection{Constructing the basic coupling}
Let $\cal{C}=(E_{x,k}, B_{x,k})_{x \in \Z, k \in \N^+}$ be a collection of independent random variables with $E_{x,k} \sim \dExp{1}$ and $B_{x,k} \sim \dBer{p=1-q}$. Define the times $T_{x,n} \defeq \sum\limits_{k=1}^n E_{x,k}$ also referred to as \textit{clock rings} and call a clock ring $T_{x,n}$ legal if the local constraint of the corresponding process is satisified at site $x$ and time $T_{x,n}$. We construct the East-process $\sigma \defeq (\sigma_t)_{t \geq 0}$ and the 1-sided contact process $\eta \defeq (\eta_t)_{t \geq 0}$ using $\cal{C}$ as follows: \\

For each site $x \in \Z$ at each time $T_{x,n}$:
\begin{itemize}
  \item If $B_{x,n} = 1$:
  \begin{enumerate}
  	\item If $\sigma_{T^-_{x,n}} (x-1) = 1$ update $\sigma$ to 1 at site x
  	\item If $\eta_{T^-_{x,n}} (x-1) = 1$ update $\eta$ to 1 at site x
  \end{enumerate}
  \item Else:
  \begin{enumerate}
  	\item If $\sigma_{T^-_{x,n}} (x-1) = 1$ update $\sigma$ to 0 at site x
  	\item Update $\eta$ to 0 at site x
  \end{enumerate}
\end{itemize}

\begin{notation}[Initial configurations]
Suppose we start a stochastic process $(\xi_t)_{t \geq 0}$ with state space $\Omega$ from initial configuration $\nu \in \Omega$. The resulting process will be denoted $(\xi^\nu_t)_{t \geq 0}$. 
\end{notation}
\begin{notation}[$\Omega$ and $\cal{P}(\Z)$]
Because of the natural bijection between the power set of $\Z$ and $\Omega$, we will consider configurations as both subsets of $\Z$ and elements of $\Omega$, switching between the two interpretations without explicit mention. 
\end{notation}

\subsection{Time change}\label{ssec:time_change}

In what follows we only consider contact processes with $\frac{p}{q} > \lambda_c$ where $\lambda_c$ is the critical parameter for the 1-sided contact process on $\Z$. A 1-sided contact process with rates satisfying this condition is called supercritical. The extinction time $\tau(\eta^{\{0\}}_.) \defeq \inf\{t \geq 0 \mid \eta^{\{0\}}_t = \varnothing \}$ of a supercritical 1-sided contact process satisfies $\Pr{\tau(\eta^{\{0\}}_.) = \infty} > 0$ i.e. the process survives forever with positive probability. 

\begin{definition}[Supercritical East-process]
As per the previous discussion, we call an East-process supercritical if $\frac{p}{q} > \lambda_c$. 
\end{definition}

\subsection{Monotonicity of the basic coupling}
The basic coupling has two important properties that follow immediately from its definition. First, it lets us construct both processes started from any initial configuration on the same probability space. \\

The second property is monotonicity: suppose at some time $t \geq 0$ $\eta_t \leq \sigma_t$. Then $\eta_{t+s} \leq \sigma_{t+s}\ \forall s \geq 0$. To see this note that $\eta$ updates a particular site to 1 only if $\sigma$ does too, and $\sigma$ updates a particular site to 0 only if $\eta$ does too. In particular, if $X(\xi)$ denotes the position of the front of $\xi \in \Omega$ then $X(\eta_{t+s}) \leq X(\sigma_{t+s})\ \forall s \geq 0$. 
