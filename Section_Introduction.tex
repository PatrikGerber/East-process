\section{Introduction and construction}\label{dec:introduction}
\label{sec:basic_coupling}

\begin{quote}
{\small In this section we introduce the main objects of study and construct the East and 1-sided contact processes on the same probability space using the graphical method. This construction gives rise to important properties used later that we discuss. }
\end{quote}

\subsection{Introduction}\label{ssec:introduction}
The East-process is an interacting particle system evolving with a Glauber like dynamics on the state space $\Omega \defeq \{0,1\}^\Z$. It belongs to a class of stochastic processes called kinetically constrained spin models (KCMs), with the East-process being the first of these to be studied rigorously. The process evolves as follows: at each site $x \in \Z$ the system tries to update the value of the spin at $x$ to 1 or 0 at rate $p \in (0,1)$ and $q \defeq 1 - p$ respectively. The update is accepted only if a local constraint is satisfied, which in the East-process' case is that the occupation variable at site $x-1$ must be equal to 1. Sometimes we will call elements of $\Omega$ \textit{configurations} and say a site is \textit{occupied} or \textit{infected} if its spin value is equal to 1. \\

In the sections to follow we focus on two objects of interest related to the East-process. The first one is the speed of the so-called \textit{front}. Consider an East-process started from the configuration equal to all 0 except at the origin. It is easy to see that the spins on $(-\infty, 0]$ stay frozen for all time, and infection 'spreads' to the right. A natural question to ask then is how fast this spreading of infection happens \textit{if} it happens at all. We define the front to be the rightmost infected site in the configuration at time $t$. We will show that for large enough $p$ the front of the East-process started from exactly one infection propagates at precisely linear speed. In our study of the speed of the front we will compare the East-process to a second stochastic process called the 1-sided contact process on $\Z$. The 1-sided contact process on $\Z$ has the same state space $\Omega$ and evolves as follows: each site infects its neighbour to the right at rate $p$ and \textit{recovers} i.e. sets its own spin to 0 at rate $q$. \\

The second object of interest is the mixing time of the East process when restricted to $\{ 0, 1, ..., L\}$ for some $L \in \N^+$. We will study the mixing time for the East-process on $\{ 0, 1, ..., L\}$ with the occupation of the origin fixed to be $1$, so that the evolution at site 1 is unconstrained. We go on to show that for large enough $p$ the mixing time is $\Theta(L)$\footnote{We say $f = \Theta(g)$ if there exist constants $K_1, K_2$ such that for all large enough $n \in \N$ it holds that $K_1 g(n) \leq f(n) \leq K_2 g(n)$.}. 

\subsection{Constructing the basic coupling}
Let $\scr{P} = (E_{x,k}, B_{x,k})_{x \in \Z, k \in \N^+}$ be a collection of independent random variables with $E_{x,k} \sim \dExp{1}$ and $B_{x,k} \sim \dBer{p=1-q}$. Define the times 
\[
T_{x,n} \defeq \sum\limits_{k=1}^n E_{x,k}
\]
 also referred to as \textit{clock rings} and call a clock ring $T_{x,n}$ \textit{legal} if the local constraint of the corresponding process is satisified at site $x$ at time $T_{x,n}$. We can now construct the East-process $(\sigma_t)_{t \geq 0}$ and the 1-sided contact process $(\eta_t)_{t \geq 0}$ using $\scr{P}$ as follows. 
 
 \newpage 

For each site $x \in \Z$ at each time $T_{x,n}$:
\begin{itemize}
  \item If $B_{x,n} = 1$:
  \begin{enumerate}
  	\item If $\sigma_{T^-_{x,n}} (x-1) = 1$ update $\sigma$ to 1 at site $x$. 
  	\item If $\eta_{T^-_{x,n}} (x-1) = 1$ update $\eta$ to 1 at site $x$. 
  \end{enumerate}
  \item Else:
  \begin{enumerate}
  	\item If $\sigma_{T^-_{x,n}} (x-1) = 1$ update $\sigma$ to 0 at site $x$. 
  	\item Update $\eta$ to 0 at site $x$. 
  \end{enumerate}
\end{itemize}

\begin{notation}[Initial configurations]
Suppose we start a stochastic process $(\xi_t)_{t \geq 0}$ with state space $\Omega$ from initial configuration $\nu \in \Omega$. The resulting process will be denoted $(\xi^\nu_t)_{t \geq 0}$. 
\end{notation}
\begin{notation}[$\Omega$ and $\cal{P}(\Z)$]
Because of the natural bijection between the power set of $\Z$ and $\Omega$, we will consider configurations as both subsets of $\Z$ and elements of $\Omega$, regularly switching between the two interpretations. 
\end{notation}

\subsection{Time change}\label{ssec:time_change}

In what follows we only consider contact processes with $\frac{p}{q} > \lambda_c$ where $\lambda_c$ is the critical parameter for the 1-sided contact process on $\Z$. A 1-sided contact process with rates satisfying this condition is called supercritical. The extinction time $\tau(\eta^{\{0\}}_.) \defeq \inf\{t \geq 0 \mid \eta^{\{0\}}_t = \varnothing \}$ of a supercritical 1-sided contact process satisfies $\Pr{\tau(\eta^{\{0\}}_.) = \infty} > 0$ i.e. the process survives forever with positive probability. 

\begin{notation}[Supercritical East-process]
As per the previous discussion, we call an East-process supercritical if $\frac{p}{q} > \lambda_c$. 
\end{notation}

\subsection{Domination and other properties}
The basic coupling has two important properties that follow immediately from its definition. First, it lets us construct both processes started from any initial configuration on the same probability space. The second property is domination: if at some time $t \geq 0$ $\eta_t \leq \sigma_t$ then $\eta_{t+s} \leq \sigma_{t+s},\ \forall s \geq 0$. To see this note that under the graphical construction $\eta$ updates a particular site to 1 only if $\sigma$ does too, and $\sigma$ updates a particular site to 0 only if $\eta$ does too. In particular, if $X(\cdot)$ denotes the position of the front then $X(\eta_{t+s}) \leq X(\sigma_{t+s}),\ \forall s \geq 0$. \\

Domination is what enables us to bound the East-process from below by the 1-sided contact process. The reason we might want to do this is that contact processes posess desirable qualities that KCMs in general might not. Contact processes are \textit{attractive} in the sense that if $\nu \subseteq \xi \subseteq \Z$ then $\eta^\nu_t \leq \eta^\xi_t$ for all $t \geq 0$ under the basic coupling. Moreover they are also \textit{additive}: if $\nu, \xi \subseteq \Z$ then $\eta^{\nu \cup \xi}_t = \eta^\nu_t \cup \eta^\xi_t$ for all $t \geq 0$ under the basic coupling. These qualities make contact processes more amenable to analysis than KCMs, and there is a breadth of methods and results already established. The East-process lacks both attractivity and additivity, thus the desire to compare it to the `simpler' 1-sided contact process is justified. 
