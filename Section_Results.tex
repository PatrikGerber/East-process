\theoremstyle{slimTheoremStyle}
\renewtheorem{theorem}{Theorem}

\section{Main results}\label{sec:results}

To state our first result we need to define the \textit{front}. Recall the notation used in Subsection \ref{ssec:intro_properties} to denote the rightmost site of a configuration in $\Omega$. Keeping in mind the identification of $\Omega$ and $\cal{P}(\Z)$ described in Notation \ref{not:powerset} we define
\begin{definition}
For $A \subseteq \Z$ we write $X(A) \defeq \max (A) \in \N \cup \{-\infty, \infty \}$ with $\max(\varnothing) \defeq -\infty$ for the \textit{front} of $A$. 
\end{definition} 

\begin{theorem}\label{main_thm:exponential_bounds}
If $(\eta_t)_{t \geq 0}$ is a supercritical, one-sided contact process then there exists $\alpha > 0$ such that for all $a < \alpha$ there exist constants $\gamma, C > 0$ such that for all $t \geq 0$ we have
\begin{align*}
  \Pr{X(\eta^{(-\infty, 0]}_t) < a t} &\leq C e^{-\gamma t}. 
  \intertext{Furthermore there exist constants $\gamma, C > 0$ such that for all $t \geq 0$ it holds that}
  \Pr{t < \tau(\eta^{\{0\}}_.) < \infty } &\leq C e^{-\gamma t}.  
\end{align*}
\end{theorem}

Using these results for the supercritical one-sided contact process we go on to show 

\begin{theorem}\label{main_thm:speed}
Let $(\sigma^{\{0\}}_t)_{t \geq 0}$ be an East process. Then there exist constants $v, \gamma > 0$ such that for all $t \geq 0$ we have
\begin{align*}
\Pr{X(\sigma^{\{0\}}_t) > vt} &\leq e^{- \gamma t}. \\
\intertext{Furthermore, if $(\sigma^{\{0\}}_t)_{t \geq 0}$ is supercritical then $\exists\ \alpha > 0$ such that $\forall\ a < \alpha$ there exist constants $\gamma, C > 0$ such that for all $t \geq 0$ it holds that } 
\Pr{X(\sigma^{\{0\}}_t) < at} &\leq C e^{-\gamma t}.
\end{align*}
\end{theorem}

Our third result concerns the \textit{mixing time} of the East process. To make sense of this first we need the notion of \textit{total variation distance} and \textit{distance from equilibrium}. 
\begin{definition}[Total variation distance]
Let $(\Omega, \cal{F})$ be a measurable space and $\nu, \mu$ be two probability measures on it. The total variation distance of $\nu$ and $\mu$ is defined as $\left\Vert \nu(\cdot) - \mu(\cdot) \right\Vert_{TV} \defeq \sup\limits_{A \in \cal{F}} \left| \nu(A) - \mu(A) \right|$
\end{definition}
\begin{definition}[Distance from equilibrium]\label{def:eq_distance}
Let $N \defeq (N_t)_{t \geq 0}$ be a continuous time, irreducible Markov chain taking values in a finite state space $\Omega$ and let $\pi$ be its equilibrium distribution. For $t \geq 0$ define 
\begin{equation}\nonumber
d(t) \defeq \max\limits_{x \in \Omega} \left\Vert \PrCond{N_t \in \cdot\ }{N_0 = x} - \pi(\cdot) \right\Vert_{TV} 
\end{equation}
to be the worst case total variation distance from equilibrium. 
\end{definition}

When restricted to $\{0, 1, ... L\}$, the East process with a fixed 1 at the origin is a finite, irreducible, continuous time Markov chain with equilibrium measure $\pi_L \defeq \delta_{1} \times \dBer{p}^L$. The mixing time then can be described as the time it takes to get close to the equilibrium distribution:
\begin{definition}[Mixing time of East process]
The mixing time of the East process with a fixed 1 at the origin is defined as $T^L_{mix} \defeq \inf\left\{ t \geq 0 \mid d(t) \leq \sfrac{1}{4} \right\}$. 
\end{definition}

\begin{theorem}\label{main_thm:mixing}
The mixing time of the East process satisfies $T^L_{mix} = \Theta(L)$. 
\end{theorem}

\begin{remark}
Our proof of Theorem \ref{main_thm:mixing} is based off Theorem \ref{main_thm:speed}. Theorem \ref{main_thm:speed} is known to hold for all $p \in (0,1)$, not just supercritical (see \cite[Theorem 6.1]{cancrini2008kinetically}). In this paper however we only prove it for $p > p_c$, hence if regarded as a self contained document we only show Theorem \ref{main_thm:mixing} for $p > p_c$. 
\end{remark}

\renewtheorem{theorem}{Theorem}[section]
