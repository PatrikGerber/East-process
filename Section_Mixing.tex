\section{Mixing of the East-process}
\begin{quote}
{\small In this section we prove that the mixing time of the supercritical East-process on $[0, L]$ with a 1 fixed at the origin is $\Omega(L)$}
\end{quote}

In this section we will study the evolution of the East-process restricted to the segment [0, L] where $L \in \N$. In doing so we will assume that there is a 1 fixed at the origin. However, because of the local constraint of the East-process, when studying the East-process restricted to $\{0, 1, ..., L\}$ it doesn't matter whether we 1) fix a 1 at the origin or 2) only consider East-processes started from $\{0\} \subseteq \xi \subseteq \N$; the results of the analysis will be the same. Motivated by this we make two definitions:

\begin{definition}
Define $\widetilde{\Omega} \defeq \{A \subseteq \N \mid 0 \in A \}$ to be the set of configurations that are 0 on the negatives and 1 at the origin. Similarly, for $L \in \N$ define $\Omega_L \defeq \widetilde{\Omega} \cap [0, L]$ to be the state space of the East-process on $\{0, 1, ..., L\}$ with a fixed 1 at the origin. 
\end{definition}

Another important property of the East-process is that the evolution at some site $x \in \N$ is not influenced by how the process evolves at sites to the right of $x$. More precisely, if $(\sigma_t)_{t \geq 0}$ is an East-process then for any $x \in \N$ and $t \geq 0$, $\sigma_t (x)$ is independent of the sigma algebra $\sigma \left( (E_{n,k}, B_{n,k})_{n > x, k > 0}\right)$. It is because of this that the East-process with a fixed 1 at the origin is a continuous time Markov chain when restricted to $\{0, 1, ..., L\}$. 

\subsection{Coupling time for basic coupling}
As before, let $(\sigma^\xi_t)_{t \geq 0}$ denote the East-process on $\Z$ started from initial configuration $\xi \in \Omega$, constructed using $\cal{C} = (E_{x,k}, B_{x,k})_{x \in \Z,\ k \in \N^+}$. Recall the definition of the hitting times $(\rho_i)_{i \in \N}$ from Definition \ref{def:hitting_times}.  


\begin{proposition}\label{prop:East_linear_coupling}
For each $l \in \N$ and $\xi \in \widetilde{\Omega}$ it holds that 
\begin{align}
\sigma^{\{0\}}_{\rho_l + t} \cap [0, l] &= \sigma^\xi_{\rho_l + t} \cap [0, l] &&\forall t \geq 0 \\
\intertext{Furthermore for all $\xi, \nu \in \widetilde{\Omega}$ it holds that}
\sigma^\xi_{\rho_l + t} \cap [0, l] &= \sigma^\nu_{\rho_l + t} \cap [0, l] &&\forall t \geq 0 
\end{align}
\end{proposition}

\begin{proof}
We proceed by induction. The claim clearly holds for $l=0$ since every such $\xi$ has a 1 at the origin forever. For the induction step suppose that the claim holds up to $l=n \geq 1$. Let $K$ be such that $T_{n+1, K} \defeq \sum\limits^{K}_{i=1} E_{n+1, i} = \rho_{n+1}$. Then $B_{n+1, K}=1$, and since the ring is legal, $\sigma^{\{0\}}_{\rho_{n+1}}(n) = 1$, furthermore by the induction hypothesis also $\sigma^\xi_{\rho_{n+1}}(n) = 1$ i.e. the ring is also legal for $\sigma^\xi_.$. Thus both processes update to 1 at time $\rho_{n+1}$. Now, since the $\rho_i$ are stopping times, $(\sigma^{\{0\}}_{\rho_{n+1}+t})_{t \geq 0}\ \&\ (\sigma^\xi_{\rho_{n+1} + t})_{t \geq 0}$ are two East-processes with $\sigma^{\{0\}}_{\rho_{n+1}} \cap [0, n+1] = \sigma^\xi_{\rho_{n+1}} \cap [0, n+1] $. The conclusion follows by basic coupling and the second claim is immediate. 
\end{proof}

\subsection{Results for continuous Markov chains}
We present some standard results for continuous time Markov chains that we will use throughout this section. 
\begin{definition}[Total variation distance]
Let $(\Omega, \cal{F})$ be a measurable space and $\nu, \mu$ be two probability measures on it. The total variation distance of $\nu$ and $\mu$ is defined as $\left\Vert \nu(\cdot) - \mu(\cdot) \right\Vert_{TV} \defeq \sup\limits_{A \in \cal{F}} \left| \nu(A) - \mu(A) \right|$
\end{definition}

% \begin{notation}
% For a finite Markov chain $(M_t)_{t \geq 0}$ taking values in $\Omega$ and started from $x \in \Omega$, we write $\P^x\left( M_t \in \cdot \right)$ for $\PrCond{M_t \in \cdot\ }{M_0 = x}$. 
% \end{notation}

\begin{definition}[Distance from equilibrium]\label{def:eq_distance}
Let $M \defeq (M_t)_{t \geq 0}$ be a continuous time, irreducible Markov chain taking values in a finite state space $\Omega$ and let $\pi$ be its equilibrium distribution. For $t \geq 0$ define $d(t) \defeq \max\limits_{x \in \Omega} \left\Vert \PrCond{M_t \in \cdot\ }{M_0 = x} - \pi(\cdot) \right\Vert_{TV}$ to be the worst case total variation distance from equilibrium. 
% Similarly define $\bar{d}(t) \defeq \max\limits_{x,y \in \Omega} \left\Vert \P^x\left( M_t \in \cdot \right) - \P^y\left( M_t \in \cdot \right) \right\Vert_{TV}$. 
\end{definition}

% \begin{lemma}\label{lem:d_ineq}
% For $d(t)$ and $\bar{d}(t)$ as defined in Definition \ref{def:eq_distance}, we have $d(t) \leq \bar{d}(t)$ for all $t \geq 0$. 
% \end{lemma}

% \begin{proof}
% Can be found in \cite{levin2017markov} Lemma 4.11 for discrete time, however the continuous time proof is identical. 
% \end{proof}

% \begin{proposition}\label{prop:coupling_ineq}
% Let $(\Omega, \cal{F})$ be a measurable space and $\nu, \mu$ be two probability measures on it. Then 
% \begin{align}
% \left\Vert \nu(\cdot) - \mu(\cdot) \right\Vert_{TV} \leq \inf\left\{ \Pr{X \neq Y} \mid (X, Y) \text{ is a coupling of } (\nu, \mu) \right\} 
% \end{align}
% \end{proposition}

% \begin{proof}
% See \cite{levin2017markov} Proposition 4.7. 
% \end{proof}

% \begin{theorem}\label{thm:coupling_time}
% Let $M \defeq (M_t)_{t \geq 0}$ be a continuous time, irreducible Markov chain taking values in a finite state space $\Omega$ and let $\pi$ be its equilibrium distribution. Let $(X_t, Y_t)$ be a coupling of $M$ with $X_0 = x$ and $Y_0 = y$. Let $\tau_{couple}\defeq \min\left\{t \geq 0 \mid X_t = Y_t \right\}$ be the first time the two chains meet. Then
% \begin{align}
% \left\Vert \P^x\left( M_t \in \cdot \right) - \P^y\left( M_t \in \cdot \right) \right\Vert_{TV} \leq \Pr{\tau_{couple} > t}
% \end{align}
% \end{theorem}

% \begin{proof}
% Follows by applying Proposition \ref{prop:coupling_ineq} and noting that $\Pr{\tau_{couple} > t} = \Pr{X_t \neq Y_t}$. 
% \end{proof}

% \begin{corollary}
% In the setting of Theorem \ref{thm:coupling_time} suppose that for each pair of states $x,y \in \Omega$ there is a coupling $(X_t, Y_t)$ started from $(x,y) \in \Omega^2$. Fo each such coupling let $\tau^{x, y}_{couple}$ be the first time the chains meet. Then 
% \begin{align}
% d(t) \leq \max\limits_{x,y \in \Omega} \Pr{\tau^{x, y}_{couple} > t}
% \end{align}
% \end{corollary}

% \begin{proof}
% Combine Lemma \ref{lem:d_ineq} and Theorem \ref{thm:coupling_time} to get the result. 
% \end{proof}

\begin{theorem}\label{thm:equilibrium_distance}
Let $M \defeq (M_t)_{t \geq 0}$ be a continuous time, irreducible Markov chain taking values in a finite state space $\Omega$ and let $\pi$ be its equilibrium distribution. Suppose that for each pair of states $x,y \in \Omega$ there is a coupling $(X_t, Y_t)_{t \geq 0}$ of $M$ that is started from $(x,y) \in \Omega^2$. For each of these couplings, let $\tau^{x, y}_{couple} \defeq \min\left\{t \geq 0 \mid X_t = Y_t \right\}$ be the first time the chains meet. Then it holds that 
\begin{align}
d(t) \leq \max\limits_{x,y \in \Omega} \Pr{\tau^{x, y}_{couple} > t}
\end{align}
\end{theorem}

\begin{proof}
The result can be found in \cite{levin2017markov} Corollary 5.3 for discrete time Markov chains, however all the necessary proofs work in the continuous time case without modification. 
\end{proof}

\subsection{Linear upper bound on mixing}
\begin{definition}[Mixing time]
In the setting of Definition \ref{def:eq_distance}, the mixing time of $M$ is defined as $t_{mix} \defeq \inf\left\{ t \geq 0 \mid d(t) \leq \sfrac{1}{4} \right\}$. 
\end{definition}

\begin{remark}[Mixing time of East-process on $\{0, 1, ... L\}$]
When restricted to $\{0, 1, ... L\}$, the East-process with a fixed 1 at the origin is a finite, irreducible, continuous time Markov chain with equilibrium measure $\pi_L \defeq \delta_{1} \times \dBer{p}^L$. Therefore the 'mixing time of the East-process on $\{0, 1, ... L\}$' refers to the mixing time of this restricted, finite chain. 
\end{remark}

\begin{theorem}
For each supercritical East-process with a 1 fixed at the origin, there exists a constant $K > 0$ such that for all $L \in \N$, the mixing time $T^L_{mix}$ on $\{0, 1 ... L\}$ satisfies $T^L_{mix} \leq KL$.  
\end{theorem}

\begin{proof}
By Theorem \ref{thm:equilibrium_distance} it holds that 
\begin{align}
d(t) \leq \max\limits_{\nu, \xi \in \widetilde{\Omega}} \Pr{ \tau^{\nu, \xi}_{couple} > t }
\end{align}
Where $\tau^{\nu, \xi}_{couple}$ is the first time that the East-process started from $\xi$ and $\nu$ (and thus with a fixed 1 at the origin) coincide on $\{0, 1 ... L\}$ under the basic coupling. Lemma \ref{prop:East_linear_coupling} gives $\max\limits_{\nu, \xi \in \widetilde{\Omega}} \Pr{\tau^{\nu, \xi}_{couple} > t } \leq \Pr{\rho_L > t}$. Borrowing notation from Corollary \ref{cor:lower_linear_speed} we get 
\begin{align*}
\Pr{\rho_L > KL} &\leq \Pr{X\left(\sigma^{\{0\}}_{KL}\right) < L} \\
                 &= \Pr{X\left(\sigma^{\{0\}}_{KL}\right) < \frac{KL}{K}} \\
  \intertext{Fix $K$ large enough such that $\sfrac{1}{K} < \alpha$ to get}
                 &\leq C e^{-\gamma K L}
\end{align*}
Thus for say $L \geq L'$, $\Pr{X\left(\sigma^{\{0\}}_{KL} \right) < L} \leq \sfrac{1}{4}$. This implies $d(KL) \leq \sfrac{1}{4}$ i.e. $T^L_{mix} \leq KL\ \forall L \geq L'$. 
\end{proof}

\subsection{Linear lower bound on mixing}
\begin{theorem}
For each East-process with a 1 fixed at the origin, there exists a constant $K > 0$ such that for all $L \in \N$, the mixing time $T^L_{mix}$ on $\{0, 1, ... L\}$ satisfies $T^L_{mix} \geq KL$.  
\end{theorem}

\begin{proof}
From the proof of Theorem \ref{thm:speed_upper_bound} we know that $\exists\ \gamma, v > 0$ such that 
\begin{align}
\Pr{\rho_{tv} \leq t} &\leq e^{- \gamma t} &&\forall t \geq 0 \\
\intertext{Thus we get}
\Pr{\rho_{\sfrac{L}{2}} \leq \sfrac{L}{2v}} &\leq e^{- \gamma \frac{L}{2v}} &&\forall L \in \N
\end{align}
Define the set $A_L \defeq \{\omega \in \Omega \mid \{\sfrac{L}{2}, \sfrac{L}{2} + 1, ..., L\} \cap \omega \neq \varnothing\}$ to be the set of configurations with at least one occupied site between $\sfrac{L}{2}$ and $L$. For our proof it is crucial to note that $\PrCond{\sigma^{\{0\}}_t \in A_L}{\rho_{\sfrac{L}{2}} > t} = 0$: since $\sigma^{\{0\}}$ is started from $\{0\}$, it can be occupied at some site $x \in \N^+$ only if site $x$ has been hit i.e. it can only happen on $\{ \rho_x \leq t \}$. Let $\pi_L \defeq \delta_1 \times \dBer{p}^L$ be the product Bernoulli measure on $\Omega_L$. We have
\begin{align*}
d\left(\frac{L}{2v}\right) &= \max\limits_{\xi \in \widetilde{\Omega}} \left\Vert \Pr{\sigma^\xi_{\sfrac{L}{2v}} \cap [0,L] \in \cdot} - \pi_L(\cdot)								   \right\Vert_{TV} \\
						   &= \max\limits_{\xi \in \Omega} \max\limits_{A \in \Omega_L} \left| \Pr{\sigma^\xi_{\sfrac{L}{2v}} \cap [0,L] \in A} - \pi_L(A) \right|\\
						   &\geq \left| \Pr{\sigma_{\sfrac{L}{2v}} \in A_L} - \pi_L(A_L) \right| \\
						   &= \left| \Pr{\rho_{\sfrac{L}{2}} \leq \sfrac{L}{2v}}\PrCond{\sigma_{\sfrac{L}{2v}} \in A_L}{\rho_{\sfrac{L}{2}} \leq \sfrac{L}{2v}}  - \left( 1 - q^L \right) \right| \\
						   &\geq \min\left\{ \left| 1 - q^L - e^{- \gamma \frac{L}{2v}} \right|,  1 - q^L\right\} \xrightarrow{L \rightarrow \infty} 1
\end{align*}
By the above, for say $L \geq L'$ it holds that $d\left(\frac{L}{2v}\right) > \sfrac{1}{4}$ i.e. $\frac{L}{2v} < T^L_{mix}$. Taking $K = \frac{1}{2v}$ the conclusion follows. 
\end{proof}



