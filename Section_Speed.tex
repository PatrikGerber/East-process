\section{Front propagation --- Proof of Theorem $2$}

\begin{quote}
{\small We prove linear speed for the front of the East process. The upper bound follows from classical results for Poisson point processes, while the lower bound is established by a comparison with the one-sided contact process, closely following the arguments of \cite{blondel2018front}. }
\end{quote}

\subsection{Restart argument}

\begin{theorem}\label{thm:restart_coupling}
For each $p < 1$ with $\frac{p}{1-p} > \lambda_c$ there exists a process $(\sigma_t, \eta_t)_{t \geq 0}$ taking values in $\Omega^2$ and a random variable $T$ taking values in $[0, \infty)$ such that 
\begin{enumerate}[(i)]
  \item $(\sigma_t)$ is a (supercritical) East process with rate parameter $p$ started from $\{0\}$
  \item $\forall t \geq 0$ and $\forall x \in \Z$, it holds that $\eta_t(x) \leq \sigma_t(x)$
  \item $(\eta_{T+t})_{t \geq 0}$ is a surviving one-sided contact process started from $\{0\}$
\end{enumerate}
Furthermore $T$ has exponentially decaying tail probabilities. 
\end{theorem}

\begin{proof}
Let $\{ \scr{C}^{(i)}\}_{i \in \N^+}$ be independent copies of $\scr{C}$. Denote by $\eta^{(i)}_.$ the one-sided contact process started from $\{0\}$, constructed using $\scr{C}^{(i)}$. Furthermore let $U_i \defeq \tau(\eta^{(i)}_.)$ be the extinction time of $\eta^{(i)}_.$. Note that the $U_i$ are i.i.d. and $\mu \defeq \Pr{U_1 = \infty} > 0$ by supercriticality. Define $L \defeq \min \{ i: U_i = \infty \}$ and note that $L$ has geometric distribution. Finally, let 
\[
T \defeq 
\left\{
  \begin{array}{ll}
    0                           & \mbox{if } L = 1 \\
    \sum\limits^{L-1}_{i=1} U_i & \mbox{otherwise}
  \end{array}
\right..
\]
First we show that $T$ has exponentially decaying tail probabilities. Note that since $T \geq 0$ almost surely, this is equivalent to finiteness of $\Ex{e^{s T}}$ for some $s > 0$. To see the latter holds for $T$ observe that conditional on $L$ the random variables $U_1, ..., U_{L-1}$ are i.i.d. with distribution equal to that of $U_1$ given $U_1 < \infty$. From Corollary \ref{cor:durrett} it follows that $U_1 | \{U_1 < \infty \}$ has exponentially decaying tail probabilities:
\begin{align*}
\Pr{U_1 > t\ | U_1 < \infty} = \frac{\Pr{t < U_1 < \infty}}{\Pr{U_1 < \infty}} \leq \frac{C e^{- \gamma t}}{1 - \mu}
\end{align*}
Therefore there exists $s > 0$ such that $\ExCond{e^{sU_1}}{U_1 < \infty} < \infty$, and so
\begin{align*}
\Ex{e^{sT}} &= \Ex{\ExCond{\exp\left(s\sum\limits_{i=1}^{L-1} U_i \right)}{L}} \\
            &= \Ex{\ExCond{e^{s U_1}}{U_1 < \infty}^{L-1}} < \infty \,,
\end{align*}
where finiteseness follows as $L$ has geometric distribution and so finite moment generating function for all $s \in \R$. \\
Now we construct the process $(\sigma_t, \eta_t)_{t \geq 0}$:
\begin{enumerate}
  \item Let $(\sigma^{[1]}_t, \eta^{[1]}_t)_{t \geq 0}$ be the basic coupling started from (\{0\}, \{0\}), constructed using $\scr{C}^{(1)}$. 
  \item Assuming $(\sigma^{[i]}_t, \eta^{[i]}_t)_{t \geq 0}$ has been constructed, define $(\sigma^{[i+1]}_t, \eta^{[i+1]}_t)_{t \geq 0}$ as :
  \begin{itemize}
    \item If $T_i \defeq \sum\limits^i_{j=1} U_j = \infty$ then $(\sigma^{[i+1]}_t, \eta^{[i+1]}_t)_{t \geq 0} \defeq (\sigma^{[i]}_t, \eta^{[i]}_t)_{t \geq 0}$
    \item Else, set $(\sigma^{[i+1]}_t, \eta^{[i+1]}_t)_{T_i > t \geq 0} \defeq (\sigma^{[i]}_t, \eta^{[i]}_t)_{T_i > t \geq 0}$ and let $(\sigma^{[i+1]}_t, \eta^{[i+1]}_t)_{t \geq T_i}$ be the basic coupling started from $(\sigma^{[i]}_{T_i}, \{0\})$, constructed using $\scr{C}^{(i+1)}$. 
  \end{itemize}
\end{enumerate}
Since $L$ has a geometric distribution, $L < \infty$ a.s. and we may define $(\sigma_t, \eta_t)_{t \geq 0} \defeq (\sigma^{[L]}_t, \eta^{[L]}_t)_{t \geq 0}$. As the $U_i$ are stopping times, $(\sigma_t)_{t \geq 0}$ is an East process started from \{0\}. It also follows that $(\eta_{T+t})_{t \geq 0}$ is a surviving one-sided contact process started from \{0\}. Noting that an East process started from \{0\} alwas has a 1 at the origin, it follows that $\eta_t \leq \sigma_t\ \forall t \geq 0$. 
\end{proof}

\subsection{Linear lower bound on propagation}

\begin{remark}
In the following proof the values of the constants $\gamma\ \&\ C$ change from line to line, without explicit mention. 
\end{remark}

\begin{proof}[Proof of lower bound in Theorem \ref{main_thm:speed}]
Let $(\sigma_t)_{t \geq 0},\ (\eta_t)_{t \geq 0}$ and $T$ be as in Theorem \ref{thm:restart_coupling}. Since $\eta_{T + .}$ survives, by Corollary \ref{cor:durrett} $\exists\ \alpha > 0$ such that for all $a < \alpha$ there exist $\gamma, C > 0$ satisfying 
\begin{align}\label{eqn:surviving_exp_decay}
  \Pr{X(\eta_{T+t}) < at} &\leq C e^{- \gamma t}  &&\forall t \geq 0. 
\end{align}
By Theorem \ref{thm:restart_coupling} part (ii) we know that $\eta_t \leq \sigma_t$ for all $t \geq 0$, so that for $c \in (0,1)$ we have
\begin{align*}
  \Pr{X(\sigma_t) < at,\ 0 \leq T \leq ct} &\leq \Pr{X(\eta_t) < at,\ (1-c)t \leq t-T \leq t} \\
                                           &=    \Pr{X(\eta_{T + (t - T)}) < at,\ t \leq \frac{t-T}{1-c} \leq \frac{t}{1-c}} \\
                                           &\leq \Pr{X(\eta_{T + (t - T)}) < \frac{a(t - T)}{1-c},\ t \leq \frac{t-T}{1-c} \leq \frac{t}{1-c}} \\
                                           &\leq \sup\limits_{u \in [(1-c)t,t]} \Pr{X(\eta_{T + u}) < \frac{au}{1-c}}. \\
  \intertext{For $c$ small enough such that $\frac{a}{1-c} < \alpha$ it follows by (\ref{eqn:surviving_exp_decay}) that}
  \sup\limits_{u \in [(1-c)t,t]} \Pr{X(\eta_{T + u}) < \frac{au}{1-c}} &\leq \sup\limits_{u \in [(1-c)t,t]} C e^{-\gamma u}  = C e^{-\gamma (1-c) t} \\
                                                                       &= C e^{-\gamma t}. \\ 
  \intertext{To get the conclusion observe that}
    \Pr{X(\sigma_t) < at}  &\leq \Pr{X(\sigma_t) < at,\ 0 \leq T \leq ct} + \Pr{T > ct} \\
                           &\leq C_1 e^{-\gamma_1 t} + C_2 e^{-\gamma_2 t} \leq C e^{-\gamma t}. 
\end{align*}
\end{proof}

\subsection{Finite speed of propagation}

\begin{definition}\label{def:hitting_times}
For the East process $(\sigma^{\{0\}}_t)_{t \geq 0}$ define $\rho_l \defeq \min\{ t \geq 0 \mid \sigma^{\{0\}}_t(l) = 1\}$ for each $l \in \N$ to be the first time that the front reaches site $l$. 
\end{definition}

Note that $0 = \rho_0 \leq \rho_1 \leq \rho_2 \leq \rho_l < \infty$ for all $l \in \N$ with finiteness following from at least linear propagation of the front since $\Pr{\rho_l = \infty} = \lim_{t \rightarrow \infty} \Pr{\rho_l > t} \leq \lim_{t \rightarrow \infty} \Pr{X\left( \sigma_t\right) < l} = 0$. We wish to bound the quantity $\Pr{\rho_l \leq t}$ to control the speed of the front since  $\{X \left( \sigma_t \right) \geq l \} \subset \{\rho_l \leq t \}$. \\

Consider the stopping times $\tau_x$ for each site $x \in \N$ defined as $\tau_{x+1} \defeq \min\{ T_{x+1, k} \mid T_{x+1, k} > \tau_x\ \text{ and } k \in \N^+ \}$ with $\tau_0 = 0$. Define the process $M_t \defeq |\{x \geq 1 \mid \tau_x \leq t \}|$ to be the process counting the number of $\tau_x$ that have occured up to time t. By repeated application of the strong Markov property it follows that the process $(M_t)_{t \geq 0}$ is in fact a Poisson process of rate 1. Define $F(M, t)$ to be the event that there is an increasing sequence of rings starting at site 1 and ending at site $M$ in the time interval $[0, t]$. By the definition of $(M_t)_{t \geq 0}$ it follows that $\Pr{F(M, t)} = \Pr{M_t \geq M}$. This, combined with the following standard result gives us the desired upper bound. 

\begin{lemma}\label{lem:chernoff}
Let $X \sim \dPoi{\lambda}$ be a Poisson random variable with mean $\lambda > 0$. Then 
\begin{align*}
\Pr{X \geq x} &\leq \frac{e^{-\lambda}(e \lambda)^x}{x^x} &&\forall x > \lambda. 
\end{align*} 
\end{lemma}

\begin{proof}
The result follows by a Chernoff bound argument. For all $t > 0$ we have
\begin{align*}
\Pr{X \geq x} &\leq \frac{\Ex{e^{tX}}}{e^{tx}} = \exp\left( \lambda e^t - \lambda - tx \right). 
\end{align*} 
The minimum occurs at $t = \log\left( \frac{x}{\lambda} \right)$, giving the result. 
\end{proof}

We are now ready to prove the linear upper bound on the speed of the East process.

\begin{remark}
In the following proof we omit the use of the floor function $\floor{\cdot}$ for clarity of notation, treating non-integer values as integers. It is however clear that the proof could be adapted to be precise. 
\end{remark}

\begin{proof}[Proof of upper bound in Theorem \ref{main_thm:speed}]
Let $v > 1$. Using Lemma \ref{lem:chernoff} and the discussion before it the calculation of an upper bound becomes straightforward:
\begin{align*}
\Pr{X(\sigma_t) > vt} &\leq \Pr{\rho_{vt} \leq t} = \Pr{M_t \geq vt} \\
                      &\leq \frac{e^{-t}(e t)^{vt}}{(vt)^{vt}} = \exp\left( -t + vt(\log\left(\sfrac{t}{vt}\right) + 1)\right) \\
                      &= \exp\left( -t + vt (1 - \log(v))\right) \leq \exp(vt (1 - \log(v)))
\end{align*}
To conclude take $v > e$ and $\gamma = - v (1 - \log(v))$. 
\end{proof}




